\chapter{Conslusion and Future Work}

\section{Conclusion}
In this disseration, we have presented a principle way to synthesize locomotion of humans and animals. Our algorithms can control characters of different morphologies to move efficiently and robustly in complex physically-simulated environments and to achieve challenging tasks. The key components of our algorithms are a set of powerful computational tools, including physical simulation and controller optimization. Although combining simulation and optimization is not a novel idea in motion synthesis, in contrast to the prior work, we examine and identify those commonly-used simplifications that can affect the quality of the motions. We eliminate these simplifications by designing new simulation and optimization techniques. Our simulators are faster, more stable and more accurate. Our optimizator can search a higher-dimensional space with both continuous and discrete variables, which may lead to better optimal solutions. These computational tools make it possible to study a more diverse set of motions in nature than were previous possible in the character animation literature.

Chapter 3 described computational tools to study the diversity of swimming motions for aquatic creatures with different body shapes. This is made possible by an accurate swimming simulation and a powerful evoluationary optimization. Compared to the simplified fluid model, our swimming simulation solves the Navier-Stokes equations. It can capture important features of water, including incompressibility and vortices, that affect swimming strategies. In contrast to the traditional alternating two-way coupling technique, our simulation solves the dynamic equations of fluids and articulated rigid bodies simultaneously. This increases the numerical stability and drastically speed up the computation. Simulating the hydrodynamic environment with Navier-Stokes equations introduce new challenges to the classical optimization algorithms. We demonstrated that CMA works well in this scenario. As a result, our algorithms can discover the most efficient swimming gait for a given creature autonumous, without any human intervention. Our results showed that the synthesized swimming motions agree well with those employed by real aquatic animals.

Chapter 4 presented computational tools to study locomotion of soft body characters without skeleton support. We developed a muscle model that worked seamlessly with the state-of-the-art FEM simulation for soft materials. This muscle model is inspired by muscle structures found in real soft body animals. It lowers the dimensionality of the control space and ensures that the overall motion is coordinated. We demonstrated how to use finite-horizon trajectory optimization to control the locomotion. The key to our success is to identify that the widely-used simplification that separates contact planning with controller optimization is not good enough to achieve a stable locomotion. To solve this problem, we formulated a QPCC and developed an efficient solver. Consequently, effecitve control strategies and natural locomotion emerge autonumously from the optimization solution.

Chapter 5 demonstrated computational tools to study agile human motions on a bicycle. We developed the first reinforcement learning algorithm that allows a virtual human character to learn bicycle stunts in a physically simulated environment. The algorithm is so efficient that most of stunt actions are learned in hours, which is even faster than the best human stunt bikers. An important lesson we have learned in this work is that it is difficult to design a good controller parametrization manually, especially for challenging locomotion tasks. The common practice of using a fixed policy parameterization tuned by users can severely limit the power of policy search algorithms. We eliminated this restriction by using NEAT, an algorithm that can simulnaneously optimize both the parametrization and the parameters of a neural network. Eventually, the virtual character learned to perform a wide variety of stunts automatically, without the tedious manual tuning of controller parametrizations.

Chapter 6 explored an efficient method to develop humanoid robot controllers for the tasks of rising from a leaning/sitting/kneeling position to an erect stance. We built an accurate physical simulation, optimized controllers in the simulation and transfered the controller to a real robot. We investigated several factors that lead to the reality gap and demonstrated in several cases that this gap may be crossed with an improved physical simulation. We perform iterative simulation calibration using data collected from robot experiments. After a small number of iterations, the controller designed in a simulation can be successfully transferred to the robot. This work shows that it is possible to apply the computational tools that were developed for character animations to design robotic controllers. It is an important milestone towards a fully automatic computational framework that can design the next-gen robots with extensive agility and maneuverability.

\section{Future Work}

The work presented in this dissertation opens the door to many promising directions for future work. Some of these future work can be accomplished in the short term by combining existing algorithms to study locomotion that is not investigated in this disseration. For example, one interesting direction is to study swimming motions of soft body animals. Studying soft body locomotion in water may have more fundamental impacts than articulated swimming creatures (Chapter 3) or soft body locomotion on land (Chapter 4). Most of the soft body animals on our planet, such as squid, octopus, seaslug and jellyfish, reside in oceans and they present more diversity in terms of swimming gaits. Their flexible body shapes could enable more efficient propulsion mechansims than those of animals with rigid skeletons. For example, a jellyfish was found as ocean's most efficient swimmer \cite{Gemmell:2013}. In addition, the surface of aquatic animals with skeletons is still highly deformable due to the presence of skins, ligament and muscles. It is more appropriate to model them as soft bodies rather than as articulated rigid bodies, as we did in Chapter 3. I believe that it would be promising to combine the approaches in Chapter 3 and 4 to synthesize swimming motions for soft body characters.

A medium term of future work is to use deep neural networks in reinforcement learning. Although NEAT frees us from laborious manual tuning of neural network structure in Chapter 5, we still need to specify the state and the action space for each task. For example, to keep balance on a bicycle, states should include the center of mass or the bicycle leaning angle. However, they may not carry over to a different task. Manually designing states and actions would not scale to more sophisticated characters, more complicated environments, or more challenging tasks. The way that we use these hand-engineered states and actions in reinforcement learning today is analogue to using HoG or SIFT features in computer vision a few years ago. Recent advance in computer vision has shown promising results, which use deep neural networks, such as autoencoder \cite{Vincent:2008} or Restricted Boltzmann machine \cite{Hinton:2012}, to learn features automatically. I believe that an important next step for reinforcement learning in character animation is to employ similar techniques to automatically discover important features for different locomotion tasks.

In the long run, the fast development of numerical algorithms and computational power will enable radical increase in efficiency and accuracy of physical simulations. The improved simulations will shrink the the Reality Gap rapidly, which will make it much easier to transfer controllers from the simulation to the real world. As a result, we envision that the two separate research fields of character animation and robotics will eventually merge and the computational tools will be shared in both fields. This will inevitably trigger a fundamental revolution in robotics in the near future, which will rely heavily on the computational tools in controller design. Chapter 6 is an important step towards realizing this vision. Although it has shown promising results, we have not yet put it into a comprehensive test with more challenging tasks. One possible stress test is to use this method to transfer bicycle stunt controllers developed in Chapter 5 to a real robot. This test will provide us insights about the causes of the Reality Gap and help us to eventually cross it.




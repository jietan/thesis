\chapter{Introduction}

Most of us human can effortlessly accomplish various locomotion tasks. We can walk, run, jump and even ride on a bicycle. In addition, Mother Nature has created an amazingly diverse set of motions in the animal kingdom. For example, birds can fly in the sky, fishes can swim in the water, geccos can crawl on a verticle surface and cats can reoriented themselves to survive a fall. Understanding and synthesizing these motions will have far-reaching impacts on a wide variety of industries, including entertainment, robotics and health care. In the entertainment industry, such as movie and games, we need to study the locomotion in nature to faithfully reproduce them on the screen, or to synthesize plausible animations for imaginary creatures. Studying  locomotion in nature can help us develop robotic controllers with extensive agility and maneuverability, such as MIT Cheetah \cite{} and Big Dog from Boston Dynamics \cite{}. These robots are expected to accomplish various missions, which are dangerous for human operators, in military, transportation, exploration and rescuing. Understanding human locomotion, can guide us build comfortable and yet powerful exoskeletons that can promote gait habilitation in children, assist paralyzed person to walk or aid normal persons to achieve superhuman ability.

Although we often take our locomotion ability for granted because we can perform them so effortlessly, it is a notoriously difficult problem because locomotion involves sophisticated neuromuscular control, sensory information processing, motion planning, coordinated muscle activation, and complicated interactions between the body and its physical environment. Despite extensive study for centuries, we have not yet fully understood the underlying physics and control mechanisms of these motions. It poses a grand challenge for scientists, engineers and artists. The focus of this thesis is to present a set of powerful computational tools that facilitate the study of the locomotion of humans and animals, for the applications in computer graphics and robotics.

\section{State of the Art}

In computer graphics, the most popular techniques to synthesize motions are key frames or motion capture. Although both methods have produced realistic character animations and special effects in movies (Toy Story \cite and Avatar \cite{}), they are not scalable and generalizable ways to synthesize motions. The key-frame method requires artistic expertise and laborious manual tuning. As a result, generating animations is a time-consuming and tedious task. For example, an 100-minute animated feature film produced by Pixar typically takes more than five years of development. On the other hand, the motion capture not only requires expensive equipment and tedious postprocessing, it is difficult to reuse the recorded data in other situations. A more principled way to synthesize motions of humans and animals is to computationally mimic the natural process that have shaped our motions. Our motions are shaped through millions of year of evolution in a world that obey physical laws. The paradigm of physically-based character animation first builds physical simulation and then optimize the motions of the character in the physically-simulated environment. Using this paradigm, many natural motions, including walking \cite{}, running \cite{}, flying \cite{} and swimming \cite{} emerge automatically from the optimization solution. In addition to these basic locomotion tasks, the research in this field has developed algorithms that allow virtual characters to recover balance from unexpected perturbations, to move in different styles, to navigate through rough terrains and to demonstrate highly skillful stunts.

Similarly, we have also seen impressive robotic systems developed in the last decade. The MIT Cheetah can run up to 20 miles per hour and jump over obstacles. Other quadrupedal robots from Boston Dynamics, Big Dog and Little Dog, can walk robustly in adversary environments, including icy or rocky terrains. The soft body robots can take advantage of their flexible bodies to navigate narrow and unstructured spaces. The humanoid robots, such as Petman and Asimo, are able demonstrate a repertoire of locomotion skills, including walking, running, dancing and climbing stairs. These amazing advance in robotics are largely attributed to the use of computational tools, including simulation, optimal control and reinforcement learning. Although these coputational tools are becoming popular and have automated some portions of the development process, the task of locomotion controller design are still limited to experts and relies on tedius manual tuning.

Despite the impressive achievements in computer graphics and robotics, the gracious, agile and diverse motions of the real creatures are far from being reproduced. Synthesizing realistic motions turns out to be a challenging problem. First of all, we are facing tremendous challenges in faithfully simulate the physical environment. For example, we have not found accurate mathematical models to describe certain physical phenomena while some dynamical systems are computationally expensive or even infeasible to simulate. More importantly, locomotion tasks often involve forceful interactions between two dynamical systems, the character itself and the surrounding environment. Since these two systems are often governed by different types of equations and simulated with different algorithms, modeling the two-way interaction between them presents a nontrivial task. An accurate and efficient physical simulator is not enough. Without proper control of muscle contactions and joint torques, a virtual character cannot move in a meaningful way to achieve an intended goal. Finding a good control mechanism poses a different set of challenges. One of the challenges is under-actuation. The movement of the center of mass (COM) can only be achieved indirectly through carefully planned interactions, such as contact, with the environment. Balance is another big challenge in locomotion. Humans and many animals are inherently unstable because their COM are above the supporting feet. The characters that we wish to control may have a large number of muscles (actuators), which need to be activated in a coordinated fashion. This results in a high dimensional nonconvex optimization problem, which can be computationally infeasible. Motion optimizing with physical constraints introduces additional difficulties. For example, the state-of-the-art contact models can invalidate the gradient, which is essential in modern optimization solvers.

Generally speaking, controlling high-dimensional dynamic systems, governed by highly nonlinear differential equations and coupled through complex mechanisms, is considered a nearly unsolvable problem. As a result, most of the prior research make simplifications on simulation models and optimization algorithms to make the computation tractable. However, many of these simplifications were made without considering the optimality of the control problems. In this thesis, we will investigate some of these simplifications and develop novel algorithms for those components that should not be simplified. 


\section{Thesis Overview}

All locomotion tasks presented in this thesis involve complex physical environments. Instead of simplifying the physical model or the optimization as in prior works, we develop new algorithms to improve both components.

Our approach follows the general framework of physically-based character animation (Figure ~\ref{fig:overview}), which includes two components: physical simulation and motion optimization. The simulation models the physics of the character, the environment and the interaction between them. The optimization searches for a controller that allows the character to fulfill user-specified locomotion tasks. 


Why should we care about complex environment?
challenges at each of them and combining them
organization of the thesis and the relation between different projects


\section{Contributions}

A stable simulation of two-way coupling between fluids and articulated rigid bodies

A finite element simulation with a muscle model for soft body animals

A optimizatoin solver for quadratic program with complimentarity conditions

A reinforcement learning framework that searches both the parametrization and the parameters of a policy

A systemamic way to transfer controllers from simulated characters to real robots


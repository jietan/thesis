\chapter{Introduction}

Mother natures creation, examples.
Why do we need to understand and synthesize these motions? Their applications.
Where are we? What are the commonly used method?

\section{State of the Art}

compared to keyframes, motion capture, physically-based character animation is better and scalable, but there are a lot of challenges


unknown high dimensional control (local minima, huge space)
complicated physics (slow simulation, numerical issues, stability, inaccurate simulation)
underactuation -> discrete contact (no gradient info)
inherent unstable

show some of state-of-the-art work in walking, flying, swimming, running, parkuring and so on. emphsize that most of the current works are in relatively simple environment

physically-based character animation, relationship with the robotics, show some most recent development in robotics, big dog, petman, MIT cheetah, swimming robot, etc. Emphasize the mannual work behind these robots.


\section{Approach}
Why should we care about complex environment?
Common elements: physical simulation and motion optimization
challenges at each of them and combining them

organization of the thesis and the relation between different projects


\section{Contributions}

A stable simulation of two-way coupling between fluids and articulated rigid bodies

A finite element simulation with a muscle model for soft body animals

A optimizatoin solver for quadratic program with complimentarity conditions

A reinforcement learning framework that searches both the parametrization and the parameters of a policy

A systemamic way to transfer controllers from simulated characters to real robots

